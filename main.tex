\documentclass{article}

\title{Dynamically Scheduled High-Level Synthesis: a Syntheesis}
\author{Lana Josipovi\'c, Radhika Ghosal and Paolo Ienne\\ Synthesis by Aur\`ele Barri\`ere}
\date{November 4th, 2018}

\begin{document}
\maketitle

\section{Introduction}

Context: FPGAs are used widely. Performance is key.
Broader applications (datacenters).

Most used hing is HLS to generate circuits. Most of them static schedules.
Everything is fixed at synthesis-time.

HLS is static. Static means we can't fully exploit parallelism.
Some dynamic information should help.

Parallel with VLIW and superscalar. Solution derives from VLIW tecniques? [28,37].
VLIW, and HLS is really successful for regular programs.

Result is new tradeoff between performance and circuit complexity.
We also want minimal programmer effort.

\section{Dynamic Scheduling}
Give examlpe where it's needed.

\section{Elastic Circuits}

% Cortadella's paper
In [?], Cortadella et al define \textit{Elastic Circuits}, a new scheme for latency-insensitive designs.
Its primary motivation is to create a solution for circuits that might deal with variable delays, for modularity or scalability purposes for instance.
Indeed, if the delay between two connected functional units is not known, then we need to enforce synchronous behaviour: the receiver should wait until the data produced by the sender has been computed, and the sender should wait until the receiver is ready to accept it.
These new circuits implement the SELF protocol, where each unit is equipped with a \textit{Valid} and a \textit{Stop} Signal. When the sender's valid signal is true (meaning that its stored data is valid), and the receiver's stop signal is false (meaning that it can accept new data), then a data transfer occurs. Otherwise, the transfer is delayed. These signals are computed dynamically, and the protocol adds only a small overhead. The result is a circuit that enforces synchronicity, can be obtained automatically and proved to be correct by construction.

FIGURE?

Inspired by this definition, and [?], which brought elastic circuits to CGRAs, the authors suggest that using elastic circuits woul allow for a dynamic schedule in High-Level Synhesis.




Definition,
Synthesis,
Registers,
Memory, ...

\section{Evaluation}

\section{Conclusion}

\end{document}
